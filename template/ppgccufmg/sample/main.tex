% INFORMAÇÃO SOBRE A VERSÃO DESSE DOCUMENTO
% VERSÃO DA CLASSE ppgccufmg 1.44 (de 2011-04-25) - http://vilarneto.com/ppgccufmg
% VERSÃO ORIGINAL: http://www.dcc.ufmg.br/pos/alunos/modelodisstese.php
% VERSÃO COMENTADA e EXEMPLIFICADA: http://www.dcc.ufmg.br/~mirella @ janeiro/2012

\documentclass[msc,proposal,hideall,showtitle,showcover,showtoc]{ppgccufmg}  % [phd] se for doutorado
                                % [phd,project] para proposta de tese 
 				                        % [msc] se for mestrado

\usepackage[brazil]{babel}      % se o documento for em português, OU
%\usepackage[english]{babel}    % se o documento for em inglês
\usepackage[utf8]{inputenc}   % Dá suporte para caracteres especiais como acentos e cedilha
\usepackage[T1]{fontenc}        % Lê a codificação de fonte T1 (font encoding default é 0T1).
\usepackage{type1ec}
\usepackage{graphicx}						% define o comando \includegraphics para a inclusão de figuras
\usepackage[a4paper,
  portuguese,
  bookmarks=true,
  bookmarksnumbered=true,
  linktocpage,
  colorlinks,
  citecolor=black,
  urlcolor=black,
  linkcolor=black,
  filecolor=black,
  ]{hyperref}
\usepackage[square]{natbib}    % permite citações naturalmente
                               % integradas ao texto
\usepackage{multirow}					 % permite fazer tabelas com múltiplas linhas

\hyphenation{re-qui-si-to % para separar corretamente 
   }

\usepackage[usenames,dvipsnames]{xcolor}  % para poder usar os nomes das cores
%cria um novo comando para comentários em vermelho no meio do texto
\newcommand{\redcomment}[1]{\textcolor{red}{\textbf{\textit{#1}}}} %M
%cria um novo comando para Dica I: em azul itálico
\newcounter{dicas}
\renewcommand{\equation}{\Roman{dicas}}
\setcounter{dicas}{1}
\newcommand{\dica}[1]{\noindent\textcolor{blue}{\textbf{\textit{Dica \Roman{dicas}: #1}}}\addtocounter{dicas}{1}} %M
%cria um novo comando para Português I: em verde itálico
\newcounter{portugues}
\renewcommand{\equation}{\Roman{dicas}}
\setcounter{portugues}{1}
\newcommand{\portugues}[1]{\noindent\textcolor{Maroon}{\textbf{\textit{Português \Roman{portugues}:}} #1}\addtocounter{portugues}{1}} %M

% novo ambiente para o exemplo
\newenvironment{Exemplo}
{%before  
   \def\FrameCommand{
      \hspace{30pt}                   % horizontal distance from left margin
      {\color{Gray}\vrule width 2pt}  % color of left rule
      {\color{White}\vrule width 6pt} % color of space between rule and text
      % \colorbox{Gray}								% background color
   }%
   \MakeFramed{\advance\hsize-\width\FrameRestore}%
   \noindent 
   \begin{adjustwidth}{}{7pt}%
   \vspace{6pt}% 
   \setstretch{.9} % "altura" de cada linha
}
{%after
   \vspace{6pt}
   \end{adjustwidth}
   \endMakeFramed
}

%cria um novo comando para um formato diferente de texto
\newcommand{\ponto}[1]{\vspace{.5cm}\noindent\textbf{#1}.} %M

\sloppy %M

\begin{document}

% O comando a seguir, \ppgccufmg, provê todas as informações relevantes para a
% classe ppgccufmg. Por favor, consulte a documentação para a descrição de
% cada chave.


% Um exemplo para documentos em português é apresentado a seguir:
\ppgccufmg{
  title={Title},
  authorrev={Surname, First Name},
  cutter={M1234x}, % INFORMAÇÃO QUE VAI NA FICHA CATALOGRÁFICA
  cdu={100.0*01.10},  % Define o identificador CDU do documento, fornecido pela Secretaria do Curso.
  university={Universidade Federal de Minas Gerais},
  course={Ciência da Computação},
  address={Belo Horizonte},
  date={2012-01}, % ANO-MÊS  DA DEFESA
  keywords={Modelo de texto, PPGCC/UFMG, Latex}, %Define as palavras-chave que deverão constar na Ficha Catalográfica,
%   separadas por vírgulas.
  advisor={Advisor's name},
%  approval={img/approvalsheet.eps},
%  abstract=[brazil]{Resumo}{resumo}, %resumo.tex
%  abstract=[english]{Abstract}{abstract}, %abstract.tex
  %abstract=[brazil]{Resumo Estendido}{resumoest}, %resumoest.tex
  %dedication={dedicatoria},
  %ack={agradecimentos},
%  ack=[Acknowledgments]{ack},
  %epigraphtext={Não existe objetivo alcançado sem trabalho realizado}{Mirella M. Moro},
 }
 %OUTRAS SUGESTÕES DE EPIGRAPH:
  %{Escolhe um trabalho de que gostes \\e não terás que trabalhar nem um dia na tua vida.}{Confúcio}
  %{O único modo de escapar da corrupção causada pelo sucesso é continuar trabalhando.}{Albert Einstein},
  %{Que ninguém se engane, só se consegue a simplicidade através de muito trabalho.}{Clarice Lispector},
  %{Um sonho sonhado sozinho é um sonho. Um sonho sonhado junto é realidade.}{Raul Seixas},
  %{Para o trabalho que gostamos levantamo-nos cedo e fazemo-lo com alegria.}{William Shakespeare},
  %{Pensar é o trabalho mais difícil que existe. Talvez por isso tão poucos se dediquem a ele.}{Henry Ford},
  %{O prazer no trabalho aperfeiçoa a obra.}{Aristóteles},
  %{O caminho batido não leva a novas pastagens}{Indira Ghandi},
  %{Uma jornada de mil milhas começa com um único passo.}{Lao-Tzu},
  %{É sempre o aventuroso que consegue grandes coisas}{Montesquieu},
  

% Um exemplo para documentos em inglês é apresentado a seguir (lembre-se de
% usar \usepackage[english]{babel}):
%\ppgccufmg{
%  title={Protocol for Error-Verification inside\\Totally Error-Free
%    Networks},
%  authorrev={da Camara Neto, Vilar Fiuza},
%  cutter={D1234p},
%  cdu={519.6*82.10},
%  university={Federal University of Minas Gerais},
%  course={Computer Science},
%  portuguesetitle={Protocolo para Verificação de Erros\\em Redes Totalmente
%    Confiáveis},
%  portugueseuniversity={Universidade Federal de Minas Gerais},
%  portuguesecourse={Ciência da Computação},
%  address={Belo Horizonte},
%  date={2008-03},
%  advisor={Adamastor Pompeu Setúbal},
%  approval={img/approvalsheet.eps},
%  abstract=[brazil]{Resumo}{resumo},
%  abstract={Abstract}{abstract},
%  dedication={dedicatoria},
%  ack={agradecimentos},
%  epigraphtext={Truth and lie are opposite things.}{Unknown},
%}

% VOCÊ PODE DIVIDIR O SEU TEXTO EM VÁRIOS ARQUIVOS, por exemplo, um para cada seção principal do seu trabalho: introducao.tex, relacionados.tex, metodologia.tex, experimentos.tex, conclusao.tex

\chapter{Introduction} \label{chap:introduction}

Here goes the introduction. You can use \\cite\{\} \cite{112493} or \\citep\{\} \citep{112493}.

% Aqui vem a parte da bibliografia: use o comando \ufmgbibliography indicando
% apenas o nome do arquivo .bib (sem a extensão).
\ppgccbibliography{bibfile} % ARQUIVO CONTENDO A BIBLIOGRAFIA

%\input{apendice} % ARQUIVO CONTENDO OS APÊNDICES : OPCIONAL
%\input{anexo} % ARQUIVO CONTENDO OS ANEXOS: OPCIONAL

\end{document}
